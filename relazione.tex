\documentclass{article}

\usepackage{amsmath}
\usepackage[italian]{babel}
\usepackage[utf8]{inputenc}

\title{LCDP - Codici correttori}
\date{14 febbraio 2018}
\author{Saftoiu Vlad Alexandru}

\begin{document}
	\pagenumbering{gobble}
	\maketitle
	\newpage
	\pagenumbering{arabic}

	\tableofcontents
	\newpage

	\section{Traccia}
	\begin{enumerate}
		\item introduzione/descrizione LDPC (e brevissima storia?)
		\item check matrix sparsa 
		\item sum-product algorithm
		\item factor graph
		\item esempi concreti di impiego di codici correttori LDPC
	\end{enumerate}
	\section{Introduzione}
	Un codice LDPC è un codice ottimo con una buona distanza, a patto di riuscire a costruire un decoder efficiente che, dato l'output $\textbf{r}$ sul canale C, individua la codeword $\textbf{t}$ con la probabilità $P(\textbf{r}|\textbf{t})$ maggiore. Decodificare un codice LDPC è un problema NP-completo, un approcio che possiamo seguire per ottenere un decoder è dato dall'utilizzo dell'algoritmo somme-prodotti a scambio di messaggi.
	\section{Algoritmo somme-prodotti a scambio di messaggi}
Obiettivo: trovare $\textbf{x}$ che massimizza $P^*(\textbf{x})=P(\textbf{x})1[\textbf{Hx} = \textbf{z}]'$.

Anche conosciuto come \textit{propagation-belief algorithm}, è un algoritmo utilizzato per fare inferenza sulle strutture ad albero (ed in maniera approssimata anche sui grafi) calcolando le probabilità marginali di un modello grafico con N variabili $\bar{x} = (x_1,x_2, x_N)$ a valori su un alfabeto finito $\mathcal{X}$.
	\subsection{Factor graph}
	Un \textit{factor graph} è un grafo bipartito che rappresenta la fattorizzazione di una funzione, in particolare viene utilizzato per rappresentare i fattori di una distribuzione di probabilità. In un \textit{factor graph} un fattore che è $0$ oppure $1$ viene chiamato \textit{constraint}.
\end{document}